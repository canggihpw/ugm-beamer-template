\usetheme{ugm1}

\title{Introduction to Beamer}
\subtitle{A short introduction to latex beamer for presentation}
\author{Your Name}
\date{\today}

\begin{document}

% Generate title page
\maketitle

% Generate section page
\section{Introduction}
\begin{frame}
    \sectionpage
\end{frame}

% Generate slide
\begin{slide}{What is Beamer?}
    \begin{itemize}
        \item A powerful LaTeX package for creating presentations.
        \item Offers a wide range of themes and customization options for a professional look.
        \item Integrates seamlessly with other LaTeX packages for advanced content inclusion (e.g., figures, equations).
      \end{itemize}
\end{slide}

\begin{slide}{Key Features of Beamer}
    \begin{itemize}
        \item Easy to use with familiar LaTeX syntax.
        \item Creates high-quality presentations with various themes.
        \item Supports animations and transitions for dynamic presentations.
        \item Integrates well with multimedia elements (images, videos).
  \end{itemize}
\end{slide}

% Generate section page
\section{Getting Started}
\begin{frame}
    \sectionpage
\end{frame}

\begin{slide}{Basic Steps with Beamer}
  \begin{itemize}
    \item Start with the beamer documentclass declaration.
    \item Define title, author, and content using Beamer specific commands.
    \item Utilize frames for individual slides with their content.
    \item Compile the code with a LaTeX compiler (e.g., pdflatex).
  \end{itemize}
\end{slide}


% Generate quote page
\quoteslide{"Anyone who stops learning is old, whether at twenty or eighty. Anyone who keeps learning stays young." - Henry Ford}


\end{document}